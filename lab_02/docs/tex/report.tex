\documentclass[12pt]{report}
\usepackage[utf8]{inputenc}
\usepackage[russian]{babel}
%\usepackage[14pt]{extsizes}
\usepackage{listings}
\usepackage{graphicx}
\usepackage{amsmath,amsfonts,amssymb,amsthm,mathtools} 
\usepackage{pgfplots}
\usepackage{filecontents}
\usepackage{float}
\usepackage{indentfirst}
\usepackage{eucal}
\usepackage{enumitem}
%s\documentclass[openany]{book}
\frenchspacing

\usepackage{indentfirst} % Красная строка

\usetikzlibrary{datavisualization}
\usetikzlibrary{datavisualization.formats.functions}

\usepackage{amsmath}


% Для листинга кода:
\lstset{ %
	language=c,                 % выбор языка для подсветки (здесь это С)
	basicstyle=\small\sffamily, % размер и начертание шрифта для подсветки кода
	numbers=left,               % где поставить нумерацию строк (слева\справа)
	numberstyle=\tiny,           % размер шрифта для номеров строк
	stepnumber=1,                   % размер шага между двумя номерами строк
	numbersep=5pt,                % как далеко отстоят номера строк от подсвечиваемого кода
	showspaces=false,            % показывать или нет пробелы специальными отступами
	showstringspaces=false,      % показывать или нет пробелы в строках
	showtabs=false,             % показывать или нет табуляцию в строках
	frame=single,              % рисовать рамку вокруг кода
	tabsize=2,                 % размер табуляции по умолчанию равен 2 пробелам
	captionpos=t,              % позиция заголовка вверху [t] или внизу [b] 
	breaklines=true,           % автоматически переносить строки (да\нет)
	breakatwhitespace=false, % переносить строки только если есть пробел
	escapeinside={\#*}{*)}   % если нужно добавить комментарии в коде
}


\usepackage[left=2cm,right=2cm, top=2cm,bottom=2cm,bindingoffset=0cm]{geometry}
% Для измененных титулов глав:
\usepackage{titlesec, blindtext, color} % подключаем нужные пакеты
\definecolor{gray75}{gray}{0.75} % определяем цвет
\newcommand{\hsp}{\hspace{20pt}} % длина линии в 20pt
% titleformat определяет стиль
\titleformat{\chapter}[hang]{\Huge\bfseries}{\thechapter\hsp\textcolor{gray75}{|}\hsp}{0pt}{\Huge\bfseries}


% plot
\usepackage{pgfplots}
\usepackage{filecontents}
\usetikzlibrary{datavisualization}
\usetikzlibrary{datavisualization.formats.functions}

\begin{document}
	%\def\chaptername{} % убирает "Глава"
	\thispagestyle{empty}
	\begin{titlepage}
		\noindent \begin{minipage}{0.15\textwidth}
			\includegraphics[width=\linewidth]{img/b_logo}
		\end{minipage}
		\noindent\begin{minipage}{0.9\textwidth}\centering
			\textbf{Министерство науки и высшего образования Российской Федерации}\\
			\textbf{Федеральное государственное бюджетное образовательное учреждение высшего образования}\\
			\textbf{~~~«Московский государственный технический университет имени Н.Э.~Баумана}\\
			\textbf{(национальный исследовательский университет)»}\\
			\textbf{(МГТУ им. Н.Э.~Баумана)}
		\end{minipage}
		
		\noindent\rule{18cm}{3pt}
		\newline\newline
		\noindent ФАКУЛЬТЕТ $\underline{\text{«Информатика и системы управления»}}$ \newline\newline
		\noindent КАФЕДРА $\underline{\text{«Программное обеспечение ЭВМ и информационные технологии»}}$\newline\newline\newline\newline\newline
		
		\begin{center}
			\noindent\begin{minipage}{1.1\textwidth}\centering
				\Large\textbf{  Отчет по лабораторной работе №2}\newline
				\textbf{по дисциплине <<Моделирование>>}\newline\newline
			\end{minipage}
		\end{center}
		
		\noindent\textbf{Тема} $\underline{\text{Метод Рунге-Кутта 4-го порядка при решении системы ОДУ}}$\newline\newline
		\noindent\textbf{Студент} $\underline{\text{Романов А.В.~~~~~~~~~~~~~~~~~~~~~~~~~~~~~~~~~~~~~~~~~~~~~~~~~~~~~~~~~~~~~}}$\newline\newline
		\noindent\textbf{Группа} $\underline{\text{ИУ7-63Б~~~~~~~~~~~~~~~~~~~~~~~~~~~~~~~~~~~~~~~~~~~~~~~~~~~~~~~~~~~~~~~~~~~~~~}}$\newline\newline
		\noindent\textbf{Оценка (баллы)} $\underline{\text{~~~~~~~~~~~~~~~~~~~~~~~~~~~~~~~~~~~~~~~~~~~~~~~~~~~~~~~~~~~~~~~~~~~~}}$\newline\newline
		\noindent\textbf{Преподаватель} $\underline{\text{Градов В. М.~~~~~~~~~~~~~~~~~~~~~~~~~~~~~~~~~~~~~~~~~~~~~~~~~~~~}}$\newline\newline\newline
		
	\begin{center}
		\vfill
		Москва~---~\the\year
		~г.
	\end{center}
\end{titlepage}
	

\section*{Тема работы}
Программно-алгоритмическая реализация метода Рунге-Кутта 4-го порядка точности при решении системы ОДУ в задаче Коши.\par

\section*{Цель работы}
Получение навыков разработки алгоритмов решения задачи Коши при реализации моделей, построенных на системе ОДУ, с использованием метода Рунге-Кутта 4-го порядка точности.\par

\section*{Теоретические сведения}

Опишем колебательный контур с помощью системы уравнений:

\begin{equation*}
	\begin{cases} L_k \frac{dI}{dt} + (R_k + R_p(I)) \cdot I - U_C = 0 
	\\ \frac{dU_c}{dt} = - \frac{I}{C_k}
	\end{cases}
\end{equation*}\\

Значение Rp(I) можно вычислить по формуле:\\

$Rp = \frac{l_e}{2 \pi \cdot \int_0^R \sigma(T(r))rdr} = \frac{l_e}{2 \pi R^2 \cdot \int_0^1 \sigma(T(z))dz}$\\

т. к. $z=r/R$. \\

\indent Значение T(z) вычисляется по формуле:\\

$T(z) = T_0 + (T_w - T_0) \cdot Z^m$\\

Заданы начальные параметры:\\\\
\indent $R$ = 0.35 см (Радиус трубки)\\
\indent $l_{e}$ = 12 см (Расстояние между электродами лампы)\\
\indent $L_{k}$ = 187e-6 Гн (Индуктивность)\\
\indent $C_{k}$ = 268e-6 Ф (Емкость конденсатора)\\
\indent $R_{k}$ = 0.25 Ом (Сопротивление)\\
\indent $U_{c0}$ = 1400 В (Напряжение на конденсаторе в начальный момент времени)\\
\indent $I_{0}$ = 0..3 А (Сила тока в цепи в начальный момент времени t = 0)\\
\indent $T_{w}$ = 2000 K 

\subsubsection*{Метод Рунге-Кутта четвертого порядка точности}

Имеем систему уравнений вида:
\begin{equation*}
	\begin{cases}
	u'(x) = f(x, u(x)) \\
	u(\xi) = \eta
	\end{cases}
\end{equation*}

Тогда:\newline

$y_{n+1} = y_n + \frac{k_1 + 2k_2 + 2k_3 + k_4}{6}$

$k_1 = h_n f(x_n, y_n)$

$k_2 = h_n f(x_n + \frac{h_n}{2}, y_n + \frac{k_1}{2})$

$k_3 = h_n f(x_n + \frac{h_n}{2}, y_n + \frac{k_2}{2})$

$k_4 = h_n f(x_n + h_n, y_n + k_3)$\newline

Рассмотрим обобщение формулы на случай двух переменных. Пусть дана система:

\begin{equation*}
	\begin{cases}
	u'(x) = f(x, u, v) \\
	v'(x) = \varphi(x, u, v) \\
	v(\xi) = v_0 \\
	u(\xi) = u_0 \\
	\end{cases}
\end{equation*}

Тогда:\newline

$y_{n+1} = y_n + \frac{k_1 + 2k_2 + 2k_3 + k_4}{6}$

$z_{n+1} = z_n + \frac{q_1 + 2q_2 + 2q_3 + q_4}{6}$\newline

$k_1 = h_n f(x_n, y_n, z_n)$

$k_2 = h_n f(x_n + \frac{h_n}{2}, y_n + \frac{k_1}{2}, z_n + \frac{q_1}{2})$

$k_3 = h_n f(x_n + \frac{h_n}{2}, y_n + \frac{k_2}{2}, z_n + \frac{q_2}{2})$

$k_4 = h_n f(x_n + h_n, y_n + k_3, z_n + q_3)$

$q_1 = h_n \varphi(x_n, y_n, z_n)$

$q_2 = h_n \varphi(x_n + \frac{h_n}{2}, y_n + \frac{k_1}{2}, z_n + \frac{q_1}{2})$

$q_3 = h_n \varphi(x_n + \frac{h_n}{2}, y_n + \frac{k_2}{2}, z_n + \frac{q_2}{2})$

$q_4 = h_n \varphi(x_n + h_n, y_n + k_3, z_n + q_3)$


\section*{Исходный код алгоритма}

\begin{lstlisting}[label=picard,caption=Реализация алгоритма Рунге-Кутта 4 порядка для решения системы ОДУ, language=python]
def T(z, T0, m):
	return T0 + (params.Tw - T0) * z ** m

def f(x, y, z, Rp):
	return -((params.Rk + Rp) * y - z) / params.Lk

def phi(x, y, z):
	return -y / params.Ck

def get_column(table, ind):
	return list(map(lambda x: x[ind], table))

def sigma(T):
	return interpolate(T, get_column(snd_table, 0), get_column(snd_table, 1))

def get_T0(I):
	return interpolate(I, get_column(fst_table, 0), get_column(fst_table, 1))

def get_m(I):
	return interpolate(I, get_column(fst_table, 0),  get_column(fst_table, 2))

def get_Rp(I, T0, m):
	integral = integrate.quad(lambda z: sigma(T(z, T0, m)) * z, 0, 1)
	return params.Le / (2 * math.pi * params.R ** 2 * integral[0])

def interpolate(x, x_pts, y_pts, order=1):
	return InterpolatedUnivariateSpline(x_pts, y_pts, k=order)(x)

def get_current_addition(h, coeffs, i, order):
	if i == 0:
		return 0, 0, 0
	elif i == order - 1:
		return h, coeffs.Kn, coeffs.Pn

	return h / 2, coeffs.Kn / 2, coeffs.Pn / 2

def get_next_members(current_y, current_z, coeffs):
	k_sum = 0, p_sum = 0
	
	for i in range(len(coeffs)):
		if i > 0 and i < len(coeffs) - 1:
			k_sum += 2 * coeffs[i].Kn
			p_sum += 2 * coeffs[i].Pn
		else:
			k_sum += coeffs[i].Kn
			p_sum += coeffs[i].Pn
	
	divider = 2 * (len(coeffs) - 2) + 2
	return current_y + k_sum / divider, current_z + p_sum / divider


def get_runge_kutta(x, y, z, h, Rp, order=4):
	coeffs = [RungeCoeffs(0, 0) for x in range(order)]
	
	for i in range(order):
		curr_h, y_add, z_add = get_current_addition(h, coeffs[i - 1], i, order)
		coeffs[i] = RungeCoeffs(h * f(x + curr_h, y + y_add, z + z_add, Rp), h * phi(x + curr_h, y + y_add, z + z_add))

	return get_next_members(y, z, coeffs)
\end{lstlisting}

\section*{Результаты работы программы}

На риснуке \ref{img:fst} представленны графики зависимости от времени импульса $t$: $I(t)$, $U(t)$, $R_{p}(t)$, $I(t)*R_{p}(t)$, $T_{0}(t)$ при исходных данных. Интервал: [0, 0.0008], шаг $h = 1e-6$.

\begin{figure}[H]
	\begin{center}
		\includegraphics[scale=0.5]{img/fst.jpg}
	\end{center}
	\caption{Графики зависимости от времени импульса $t$}
	\label{img:fst}
\end{figure}

На риснуке \ref{img:snd} представлен график $I(t)$, при $R_{k} + R_{p} = 0$. Интервал: [0, 0.0008], шаг $h = 1e-6$.

\begin{figure}[H]
	\begin{center}
		\includegraphics[scale=0.56]{img/snd.jpg}
	\end{center}
	\caption{График зависимости $I(t)$ при $R_{k} + R_{p} = 0$}
	\label{img:snd}
\end{figure}

На риснуке \ref{img:thd} представлен график $I(t)$, при $R_{k} + R_{p} = 200$. Интервал: [0, 0.00002], шаг $h = 1e-7$.

\begin{figure}[H]
	\begin{center}
		\includegraphics[scale=0.6]{img/thd.jpg}
	\end{center}
	\caption{График зависимости $I(t)$ при $R_{k} + R_{p} = const = 200$}
	\label{img:thd}
\end{figure}

На рисунках \ref{img:3_1} - \ref{img:3_7} представлены результаты исследования влияния параметров контура $C_{k}, L_{k} и R_{k}$ на длительность импульса $t$.

\begin{figure}[H]
	\begin{center}
		\includegraphics[scale=0.6]{img/3_1.jpg}
	\end{center}
	\caption{График зависимости $I(t)$ при начальных параметрах}
	\label{img:3_1}
\end{figure}

\begin{figure}[H]
	\begin{center}
		\includegraphics[scale=0.6]{img/3_2.jpg}
	\end{center}
	\caption{График зависимости $I(t)$ при увеличении начального значения $C_{k}$ в 2 раза.}
	\label{img:3_2}
\end{figure}

\begin{figure}[H]
	\begin{center}
		\includegraphics[scale=0.6]{img/3_3.jpg}
	\end{center}
	\caption{График зависимости $I(t)$ при уменьшении начального значения $C_{k}$ в 1.5 раза.}
	\label{img:3_3}
\end{figure}

\begin{figure}[H]
	\begin{center}
		\includegraphics[scale=0.6]{img/3_4.jpg}
	\end{center}
	\caption{График зависимости $I(t)$ при увеличении начального значения $L_{k}$ в 2 раза}
	\label{img:3_4}
\end{figure}

\begin{figure}[H]
	\begin{center}
		\includegraphics[scale=0.6]{img/3_5.jpg}
	\end{center}
	\caption{График зависимости $I(t)$ при уменьшении начального значения $L_{k}$ в 2 раза}
	\label{img:3_5}
\end{figure}

\begin{figure}[H]
	\begin{center}
		\includegraphics[scale=0.6]{img/3_6.jpg}
	\end{center}
	\caption{График зависимости $I(t)$ при увеличении начального значения $C_{k}$ в 10 раз}
	\label{img:3_6}
\end{figure}

\begin{figure}[H]
	\begin{center}
		\includegraphics[scale=0.6]{img/3_7.jpg}
	\end{center}
	\caption{График зависимости $I(t)$ при уменьшении начального значения $C_{k}$ в 10 раз}
	\label{img:3_7}
\end{figure}

\begin{itemize}
	\item увеличение $C_{k}$ приводит к увеличению длительности импульса $t$;
	\item уменьшение $C_{k}$ приводит к уменьшению длительности импульса $t$;
	
	\item увеличение $L_{k}$ приводит к увеличению длительности импульса $t$;
	\item уменьшение $L_{k}$ приводит к уменьшению длительности импульса $t$;
	
	\item увеличение $R_{k}$ приводит к увеличению длительности импульса $t$;
	\item уменьшение $R_{k}$ приводит к уменьшению длительности импульса $t$;
\end{itemize}

\section*{Ответы на вопросы}

\textbf{1. Какие способы тестирования программы, кроме указанного в п. 2, можете предложить ещё?}\\

Мы можем убрать лампу: при небольших значениях $R_{k}$ получим затузающие колебания, при больших значения $R_{k}$ получим апериодическое затухание.\\

\textbf{2. Получите систему разностных уравнений для решения сформулированной задачи неявным методом трапеций. Опишите  алгоритм реализации полученных уравнений.}

\begin{equation}
	U_{n + 1} = U_{n} + \frac{h}{2} + f(x_{n}, u_{n}) + f(x_{n + 1}, u_{n + 1}) + O(h^2)
\end{equation}

\begin{equation}
	\begin{cases}
		\frac{dI}{dT} = \frac{U - (R_{k} + R_{p}(I))I}{L_{k}}\\
		\frac{dU}{dt} = -\frac{I}{C_{k}}\\
	\end{cases}
\end{equation}

\begin{equation}
	I_{n + 1} = I_{n} + \frac{h}{2} (\frac{U_{n} - (R_{k} + R_{p}(I_{n}))I_{n}}{L_{k}} + \frac{U_{n + 1} - (R_{k} + R_{p}(I_{n + 1}))I_{n + 1}}{L_{k}})
\end{equation}

\begin{equation}
	U_{n + 1} = U_{n} + \frac{h}{2}(-\frac{I_{n}}{C_{k}} -\frac{I_{n + 1}}{C_{k}}) = U_{n} - \frac{h}{2}(\frac{I_{n} + I_{n + 1}}{C_{k}})
\end{equation}

Подставляя (\textbf{4}) в (\textbf{3}), имеем: 

\begin{equation}
	I_{n + 1} = I_{n} + \frac{h}{2L_{k}}(2U_{n} -  (R_{k} + R_{p}(I_{n}) + \frac{h}{2C_{k}})I_{n} - (R_{k} + R_{p}(I_{n + 1}) + \frac{h}{2C_{k}})I_{n+1})
\end{equation}\\

\textbf{3. Из каких соображений проводится выбор численного метода того или иного порядка точности, учитывая, что чем выше порядок точности метода, тем он более сложени требует, как правило, больших ресурсов вычислительной системы?}\\

Оценивается погрещность для частного случая вида правой части дифференциального уравнения: $\varphi (x, \nu) = \varphi(x)$\\

Так, если $\varphi (x, \nu)$ непрерывна и ограничена и ограничены и непрерывны её четвертые производные, то наилучший результат достигаем при\\

$y_{n + 1} = y_{n} + \frac{k_{1} + 2k_{2} + 2k_{3} + k_{4}}{6}$, где\\

$k_{1} = h_{n} f(x_{n}, y_{n})$\\
\indent $k_{2} = h_{n}f(x_{n} + \frac{h_{n})}{2}, y_{n} + \frac{k_{1}}{2}$\\
\indent $k_{3} = h_{n}f(x_{n} + \frac{h_{n})}{2}, y_{n} + \frac{k_{2}}{2}$\\
\indent $k_{4} = h_{n}f(x_{n} + \frac{h_{n})}{2}, y_{n} + k_{3}$\\

В случае если $\varphi (x, \nu)$ не имеет таких производных, то четвертый порядок схемы не может быть достигнут и стоит применять более простые схемы.

\bibliographystyle{utf8gost705u}  % стилевой файл для оформления по ГОСТу
	
\bibliography{51-biblio}          % имя библиографической базы (bib-файла)
	
	
\end{document}
