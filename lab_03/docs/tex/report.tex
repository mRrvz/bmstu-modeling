\documentclass[12pt]{report}
\usepackage[utf8]{inputenc}
\usepackage[russian]{babel}
%\usepackage[14pt]{extsizes}
\usepackage{listings}
\usepackage{graphicx}
\usepackage{amsmath,amsfonts,amssymb,amsthm,mathtools} 
\usepackage{pgfplots}
\usepackage{filecontents}
\usepackage{float}
\usepackage{indentfirst}
\usepackage{eucal}
\usepackage{enumitem}
%s\documentclass[openany]{book}
\frenchspacing

\usepackage{indentfirst} % Красная строка

\usetikzlibrary{datavisualization}
\usetikzlibrary{datavisualization.formats.functions}

\usepackage{amsmath}


% Для листинга кода:
\lstset{ %
	language=c,                 % выбор языка для подсветки (здесь это С)
	basicstyle=\small\sffamily, % размер и начертание шрифта для подсветки кода
	numbers=left,               % где поставить нумерацию строк (слева\справа)
	numberstyle=\tiny,           % размер шрифта для номеров строк
	stepnumber=1,                   % размер шага между двумя номерами строк
	numbersep=5pt,                % как далеко отстоят номера строк от подсвечиваемого кода
	showspaces=false,            % показывать или нет пробелы специальными отступами
	showstringspaces=false,      % показывать или нет пробелы в строках
	showtabs=false,             % показывать или нет табуляцию в строках
	frame=single,              % рисовать рамку вокруг кода
	tabsize=2,                 % размер табуляции по умолчанию равен 2 пробелам
	captionpos=t,              % позиция заголовка вверху [t] или внизу [b] 
	breaklines=true,           % автоматически переносить строки (да\нет)
	breakatwhitespace=false, % переносить строки только если есть пробел
	escapeinside={\#*}{*)}   % если нужно добавить комментарии в коде
}


\usepackage[left=2cm,right=2cm, top=2cm,bottom=2cm,bindingoffset=0cm]{geometry}
% Для измененных титулов глав:
\usepackage{titlesec, blindtext, color} % подключаем нужные пакеты
\definecolor{gray75}{gray}{0.75} % определяем цвет
\newcommand{\hsp}{\hspace{20pt}} % длина линии в 20pt
% titleformat определяет стиль
\titleformat{\chapter}[hang]{\Huge\bfseries}{\thechapter\hsp\textcolor{gray75}{|}\hsp}{0pt}{\Huge\bfseries}


% plot
\usepackage{pgfplots}
\usepackage{filecontents}
\usetikzlibrary{datavisualization}
\usetikzlibrary{datavisualization.formats.functions}

\begin{document}
	%\def\chaptername{} % убирает "Глава"
	\thispagestyle{empty}
	\begin{titlepage}
		\noindent \begin{minipage}{0.15\textwidth}
			\includegraphics[width=\linewidth]{img/b_logo}
		\end{minipage}
		\noindent\begin{minipage}{0.9\textwidth}\centering
			\textbf{Министерство науки и высшего образования Российской Федерации}\\
			\textbf{Федеральное государственное бюджетное образовательное учреждение высшего образования}\\
			\textbf{~~~«Московский государственный технический университет имени Н.Э.~Баумана}\\
			\textbf{(национальный исследовательский университет)»}\\
			\textbf{(МГТУ им. Н.Э.~Баумана)}
		\end{minipage}
		
		\noindent\rule{18cm}{3pt}
		\newline\newline
		\noindent ФАКУЛЬТЕТ $\underline{\text{«Информатика и системы управления»}}$ \newline\newline
		\noindent КАФЕДРА $\underline{\text{«Программное обеспечение ЭВМ и информационные технологии»}}$\newline\newline\newline\newline\newline
		
		\begin{center}
			\noindent\begin{minipage}{1.1\textwidth}\centering
				\Large\textbf{  Отчет по лабораторной работе №3}\newline
				\textbf{по дисциплине <<Моделирование>>}\newline\newline
			\end{minipage}
		\end{center}
		
		\noindent\textbf{Тема} $\underline{\text{ОДУ второго порядка с краевыми условиями 2 и 3 рода}}$\newline\newline
		\noindent\textbf{Студент} $\underline{\text{Романов А.В.~~~~~~~~~~~~~~~~~~~~~~~~~~~~~~~~~~~~~~~~~~~~~~~~~~~~~~}}$\newline\newline
		\noindent\textbf{Группа} $\underline{\text{ИУ7-63Б~~~~~~~~~~~~~~~~~~~~~~~~~~~~~~~~~~~~~~~~~~~~~~~~~~~~~~~~~~~~~~}}$\newline\newline
		\noindent\textbf{Оценка (баллы)} $\underline{\text{~~~~~~~~~~~~~~~~~~~~~~~~~~~~~~~~~~~~~~~~~~~~~~~~~~~~~~~~~~~~~}}$\newline\newline
		\noindent\textbf{Преподаватель} $\underline{\text{Градов В. М.~~~~~~~~~~~~~~~~~~~~~~~~~~~~~~~~~~~~~~~~~~~~~}}$\newline\newline\newline
		
	\begin{center}
		\vfill
		Москва~---~\the\year
		~г.
	\end{center}
\end{titlepage}
	

\section*{Тема работы}
Программно-алгоритмическая реализация моделей на основе ОДУ второго порядка с краевыми условиями II и III рода.\par

\section*{Цель работы}
Получение навыков разработки алгоритмов решения краевой задачи при реализации моделей, построенных на ОДУ второго порядка.\par

\section*{Теоретические сведения}

Задана математическая модель:

\begin{equation*}
	\frac{d}{dx}(\lambda(T)\frac{dT}{dx}) - 4 \cdot k(T) \cdot n_{p}^2 \cdot \sigma \cdot (T^4 - T_{0}^4) = 0
\end{equation*}\\

Краевые условия:

\begin{equation*}
	\begin{cases} x = 0, -\lambda(T(0))\frac{dT}{dx} = F_{0}.
	\\ x = l, -\lambda(T(l))\frac{dT}{dx} = \alpha(T(l) - T_{0})
	\end{cases}
\end{equation*}\\

Функции $\lambda(T)$ и $k(T)$ заданы таблицей.\\

Заданы начальные параметры:\\
\indent $n_p$ = 1.4 - коэффициент преломления\\
\indent $l$ = 0.2 см - толщина слоя\\
\indent $T_{0}$ = 300К - температура окружающей среды\\
\indent $\sigma$ = 5.668 $\cdot 10^{-12}$ Вт / ($cm^2 \cdot K^4$) - постоянная Стефана - Больцмана\\
\indent $F_{0}$ = 100 Вт / $cm^2$ - поток тепла\\
\indent $\alpha$ = 0.05 Вт / ($cm^2 \cdot K$) - коэффициент теплоотдачи\\

Выход из итераций организовать по температуре и по балансу энергии:

\begin{equation*}
	max|\frac{y^s_n - y^{s-1}_n}{y^s_n}| <= \varepsilon_{1}
\end{equation*} 

\indent для всех $n = 0, 1, ... N.$ и \\

\begin{equation*}
	max|\frac{f^s_1 - y^s_2}{f^s_1}| <= \varepsilon_{1}
\end{equation*}

где \\

\begin{equation*}
	f_{1} = F_0 - \alpha(T(l) - T_{0})
\end{equation*}

\begin{equation*}
	f_{2}  = 4n^2_p \sigma ^1_0 k(T(x))(T^4(x) - T^4_0) dx
\end{equation*}

\section*{Исходный код алгоритма}

\begin{lstlisting}[label=picard,caption=, language=python]
from collections import namedtuple
from math import pow
from scipy.interpolate import InterpolatedUnivariateSpline
from numpy import arange

import plot

Params = namedtuple('Params', 'Np l T0 Tconst, sigma F0 alpha h')

params = Params(
	1.4, 0.2, 300, 400, 5.668 * pow(10, -12), 100, 0.05, pow(10, -4)
)

fst_table = (
	(
		300, 500, 800, 1100, 2000, 2400
	),
	(
		1.36 * pow(10, -2), 1.63 * pow(10, -2), 1.81 * pow(10, -2),
		1.98 * pow(10, -2), 2.50 * pow(10, -2), 2.74 * pow(10, -2)
	)
)

snd_table = (
	(
		293, 1278, 1528, 1677, 2000, 2400
	),
	(
		2.0 * pow(10, -2), 5.0 * pow(10, -2), 7.8 * pow(10, -2),
		1.0 * pow(10, -1), 1.3 * pow(10, -1), 2.0 * pow(10, -1)
	),
)

def interpolate(x_pts, y_pts, order=1):
	return InterpolatedUnivariateSpline(x_pts, y_pts, k=order)


def p(k_t, t, n):
	return 4 * params.Np * params.Np * params.sigma * k_t(t[n]) * pow(t[n], 3)


def f(k_t, t, n):
	return 4 * params.Np *params.Np + params.sigma * k_t(t[n]) * pow(params.T0, 4)


def x_right(l_t, t, n):
	return (l_t(t[n]) + l_t(t[n + 1])) / 2


def x_left(l_t, t, n):
	return (l_t(t[n]) + l_t(t[n - 1])) / 2
	

def p_right(k_t, t, n):
	return (p(k_t, t, n) + p(k_t, t, n + 1)) / 2


def p_left(k_t, t, n):
	return (p(k_t, t, n) + p(k_t, t, n - 1)) / 2


def f_right(k_t, t, n):
	return (f(k_t, t, n) + f(k_t, t, n + 1)) / 2


def f_left(k_t, t, n):
	return (f(k_t, t, n) + f(k_t, t, n - 1)) / 2


def A(l_t, t, n):
	return (l_t(t[n]) + l_t(t[n - 1])) / 2 / params.h


def B(l_t, k_t, t, n):
	return A(l_t, t, n) + C(l_t, t, n) + 4 * params.Np * params.Np * params.sigma * k_t(t[n]) * pow(t[n], 3) * params.h


def C(l_t, t, n):
	return (l_t(t[n]) + l_t(t[n + 1])) / 2 / params.h


def D(k_t, t, n):
	return 4 * params.Np * params.Np + params.sigma * k_t(t[n]) * pow(params.T0, 4) * params.h


def get_right_conditions(l_t, k_t, t):
	K0 = x_right(l_t, t, 0) + pow(params.h, 2) / 8 * p_right(k_t, t, 0) + pow(params.h, 2) / 4 * p(k_t, t, 0)
	M0 = pow(params.h, 2) / 8 * p_right(k_t, t, 0) - x_right(l_t, t, 0)
	P0 = params.h * params.F0 + pow(params.h, 2) / 4 * (f_right(k_t, t, 0) + f_left(k_t, t, 0))

	return K0, M0, P0


def get_left_conditions(k_t, l_t, t, n):
	Kn = x_left(l_t, t, n) / params.h - params.alpha - params.h * p(k_t, t, n) / 4 - params.h * p_left(k_t, t, n) / 8
	Mn = x_left(l_t, t, n) / params.h - params.h * p_left(k_t, t, n) / 8
	Pn = -(params.alpha * params.T0 + (f_right(k_t, t, n) + f_left(k_t, t, n)) / 4 * params.h)

	return Kn, Mn, Pn


def start():
	l_t = interpolate(fst_table[0], fst_table[1])
	k_t = interpolate(snd_table[0], snd_table[1])
	t = [0 for _ in range(int(1 / params.h))]
	
	K0, M0, P0 = get_right_conditions(l_t, k_t, t)

	xi_list = [0]
	eta_list = [0]
	x_list = list()

	x = 0
	n = 0
	while x + params.h < 1:
		x_list.append(x)

		xi_list.append(C(l_t, t, n) / (B(l_t, k_t, t, n) - A(l_t, t, n) * xi_list[n]))
		eta_list.append((D(k_t, t, n) + A(l_t, t, n) * xi_list[n]) / (B(l_t, k_t, t, n) - A(l_t, t, n) * xi_list[n]))

		n += 1
		x += params.h

	Kn, Mn, Pn = get_left_conditions(k_t, l_t, t, n)

	t[n] = (Pn - Mn * xi_list[n]) / (Kn + Mn * xi_list[n])
	for i in range(n - 1, -1, -1):
		t[i] = xi_list[i + 1] * t[i + 1] + eta_list[i + 1]
\end{lstlisting}

\section*{Результаты работы программы}

\textbf{1. Представить разностный аналог краевого условия при $x = l$ и его краткий вывод интегро-интерполяционным методом.}

Проинтегрируем уравнение на отрезке [$X_{n - \frac{1}{2}}; x_{n}$]

\begin{equation*}
	- \int^{x_{n}}_{x_n - \frac{1}{2}} \frac{dF}{dx} dT - \int^{x_n}_{x_n - \frac{1}{2}} P(T) \cdot T^4 dT + \int^{x_{n}}_{x_{n} - \frac{1}{2}} f(t) dT = 0
\end{equation*}

Второй и третий интеграл вычислим с помощью метода трапеций:

\begin{equation*}
	F_{n - \frac{1}{2}} - F_{n} - \frac{h}{4} (p_{n} y_{n} + p_{n - \frac{1}{2}}y_{n-\frac{1}{2}}) + \frac{h}{4} (f_{n} + f_{n - \frac{1}{2}}) = 0
\end{equation*}

Зная, что: 

\begin{equation*}
	F_{n - \frac{1}{2}} = x_{n - \frac{1}{2}} \frac{y_{n - 1}}{y_n}{h}
\end{equation*}

\begin{equation*}
	F_{n} = \alpha_{n}(y_{n} - T_{0})
\end{equation*}

\begin{equation*}
	y_{n - \frac{1}{2}} = \frac{y_{n} + y_{n - 1}}{2h}
\end{equation*}

Имеем:

\begin{equation*}
	\frac{x_{n - \frac{1}{2}} y_{n - 1}}{h} - \frac{x_{n - \frac{1}{2}}y_{n}}{h} - \alpha_{n}y_{n} + \alpha_{n} T_{0} - \frac{hp_{n}y_{n}}{48} - \frac{hp_{n - \frac{1}{2}}y_{n}}{8} - \frac{hp_{n - \frac{1}{2}}y_{n - 1}}{8} + \frac{f_{n - \frac{1}{2}} + f_{n}}{4}h = 0
\end{equation*}

\begin{equation*}
	y_{n}(-\frac{x_{n - \frac{1}{2}}}{h} - \alpha_{n} - \frac{hp_{n}}{4} - \frac{hp_{n} - \frac{1}{2}}{8}) + y_{n - 1}(\frac{x_{n - \frac{1}{2}}}{h} - \frac{hp_{n - \frac{1}{2}}}{8}) = -(\alpha_{n}T_{0} + \frac{f_{n} - \frac{1}{2}}{4}h)
\end{equation*}

\clearpage
\textbf{2. График зависимости температуры $T(x)$ координаты $x$ при заданных выше параметрах.}\\

\begin{figure}[H]
	\begin{center}
		\includegraphics[scale=0.75]{img/1.png}
	\end{center}
	%\caption{Графики зависимости от времени импульса $t$}
	\label{img:1}
\end{figure}

\textbf{3. График зависимости $T(x)$ при $F_{0}$ = -10 Bт / $cm^2$. }\\

\begin{figure}[H]
	\begin{center}
		\includegraphics[scale=0.75]{img/2.png}
	\end{center}
	%\caption{Графики зависимости от времени импульса $t$}
	\label{img:2}
\end{figure}

\textbf{4. График зависимости $T(x)$ при увеличенных значениях $\alpha$ (например, в 3 раза). Сравнить с п. 2.}\\

\begin{figure}[H]
	\begin{center}
		\includegraphics[scale=0.75]{img/3.png}
	\end{center}
	%\caption{Графики зависимости от времени импульса $t$}
	\label{img:3}
\end{figure}

\textbf{5. График зависимости $T(x)$ при $F_{0} = 0$.}\\

\begin{figure}[H]
	\begin{center}
		\includegraphics[scale=0.7]{img/4.png}
	\end{center}
	%\caption{Графики зависимости от времени импульса $t$}
	\label{img:4}
\end{figure}

\textbf{6. Для указанного в задании исходного набора параметров привести данные по балансу энергии.}\\

\begin{itemize}
	\item Точность выхода $\varepsilon_{1} = 0.069$ (по температуре)
	\item Точность выхода $\varepsilon_{2} = 1.12$ (по балансу)
\end{itemize}

\section*{Ответы на вопросы}

\textbf{1. Какие способы тестирования программы можно предложить?}\\

При $F_0 > 0$ происходит охлаждение пластины, при $F_0 < 0$ нагревание. Кроме того, при увеличении показтеля теплосъема, уровень должен снижаться, а градиент увеличиваться.\\

\textbf{2. Получите  простейший разностный аналог нелинейного краевого условия при $x = l$.}\\

Аппроксимируем производную:
\begin{equation*}
	\frac{dT}{dx} = \frac{y_{N} - y_{N - 1}}{h}
\end{equation*}

Подставим в исходное уравнение:

\begin{equation*}
	-k_{N}\frac{y_{n} - y_{N - 1}}{h} = \alpha_{N} (y_{N} - T_{0}) + \varphi(y_{N})
\end{equation*}

Учтём, что $y_{N - 1} = \xi_{N}y_{N} + \eta_{N}:$

\begin{equation*}
	-k_{N}(y_{N} - \xi_{N}y_{N} + \eta_{N}) = \alpha_{N}(y_{N} - T_{0})h + \varphi(y_{N})h
\end{equation*}

Приводя подобные, получим:

\begin{equation*}
	\varphi(y_{N})h + (k_{N} + \alpha_{N} h - k_{N} \xi_{N} - k_{N} \eta_{N})y_{N} - h\alpha_{N}T_{0} = 0
\end{equation*}\\

\textbf{3. Опишите алгоритм применения метода прогонки, если при $x = 0$ краевое условие квазилинейное (как в настоящей работе), а при $x = l$, как в п. 2.}\\

Найдем начальные прогочные коэффициенты по формулам:

\begin{equation*}
	\xi = \frac{-M_{0}}{P_{0}}
\end{equation*}

\begin{equation*}
	\eta = \frac{-K_{0}}{P_{0}}
\end{equation*}

коэффициенты $M_0, P_0, K_0$ были полученые в лекции №7. Далее, находим последующие прогоночные коэффициенты:

\begin{equation*}
	\xi_{n + 1} = \frac{C_{n}}{B_{n} - A_{n}\xi_{n}}
\end{equation*}

\begin{equation*}
	\eta_{n + 1} = \frac{F_{n} + A_{n}\eta_{n}}{B_{n} - A_{n}\xi_{n}}
\end{equation*}

Из уравнения, полученного в п. 2., можем получить $y_{N}$ (решив это уравнение). По прогочной формуле можем найти все значения неизвестных $y_{N}$

\begin{equation*}
	y_{n} = \xi_{n + 1}y_{n + 1} + \eta_{n + 1}
\end{equation*}\\

\textbf{4. Опишите алгоритм определения единственного значения сеточной функции $y_p$ в одной заданной точке $p$. Использовать встречную прогонку, т.е. комбинацию правой и левой прогонок.}\\

\textbf{1. Вычислим начальные прогоночные коэффициенты:}\\

Для правой прогонки:

\begin{equation*}
	\xi = \frac{-M_0}{P_0}
\end{equation*}

\begin{equation*}
	\xi = \frac{-K_0}{P_0}
\end{equation*}

Для левой прогонки:

\begin{equation*}
	\alpha_{N - 1} = \frac{-M_{N}}{K_{N}}
\end{equation*}

\begin{equation*}
	\beta_{N - 1} = \frac{-P_{N}}{K_{N}}
\end{equation*}

\textbf{2. Найдем прогоночные коэффициенты:}\\

Для левой прогонки:

\begin{equation*}
	\xi_{n + 1} = \frac{C_{n}}{B_{n} - A_{n}\xi_{n}}
\end{equation*}

\begin{equation*}
	\eta_{n + 1} = \frac{F_{n} + A_{n}\eta_{n}}{B_{n} - A_{n}\xi_{n}}
\end{equation*}

Для правой прогонки:

\begin{equation*}
	\alpha_{n - 1} = \frac{A_{n}}{B_{n} - C_{n}\alpha_{n}}
\end{equation*}

\begin{equation*}
	\beta_{n - 1} = \frac{F_{n} + C_{n}\beta_{n}} {B_{n} - C_{n}\alpha_{n}}
\end{equation*}

\textbf{3. Левые и правые прогонки: }

\begin{equation*}
	y_{n} = \xi_{n + 1} y_{n + 1} + \eta_{n + 1}
\end{equation*}

\begin{equation*}
	y_{n} = \alpha_{n - 1} y_{n + 1} + \beta_{n - 1}
\end{equation*}

\textbf{4. Выразим $y_{p}$: }

\begin{equation*}
	y_{p - 1} = \xi_{p}y_{p} + \eta_{p}
\end{equation*}

\begin{equation*}
	y_{p} = \alpha_{p - 1} y_{p - 1} + \beta_{p - 1}
\end{equation*}

\begin{equation*}
	y_{p} = \frac{\xi_{n +1} \beta_{n} + \eta_{n + 1}}{1 - \xi_{n + 1} \alpha_ {n}}
\end{equation*}

\bibliographystyle{utf8gost705u}  % стилевой файл для оформления по ГОСТу
	
\bibliography{51-biblio}          % имя библиографической базы (bib-файла)
	
	
\end{document}
