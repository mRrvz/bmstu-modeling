\documentclass[12pt]{report}
\usepackage[utf8]{inputenc}
\usepackage[russian]{babel}
%\usepackage[14pt]{extsizes}
\usepackage{listings}
\usepackage{graphicx}
\usepackage{amsmath,amsfonts,amssymb,amsthm,mathtools} 
\usepackage{pgfplots}
\usepackage{filecontents}
\usepackage{float}
\usepackage{indentfirst}
\usepackage{eucal}
\usepackage{enumitem}
%s\documentclass[openany]{book}
\frenchspacing

\usepackage{indentfirst} % Красная строка

\usetikzlibrary{datavisualization}
\usetikzlibrary{datavisualization.formats.functions}

\usepackage{amsmath}


% Для листинга кода:
\lstset{ %
	language=c,                 % выбор языка для подсветки (здесь это С)
	basicstyle=\small\sffamily, % размер и начертание шрифта для подсветки кода
	numbers=left,               % где поставить нумерацию строк (слева\справа)
	numberstyle=\tiny,           % размер шрифта для номеров строк
	stepnumber=1,                   % размер шага между двумя номерами строк
	numbersep=5pt,                % как далеко отстоят номера строк от подсвечиваемого кода
	showspaces=false,            % показывать или нет пробелы специальными отступами
	showstringspaces=false,      % показывать или нет пробелы в строках
	showtabs=false,             % показывать или нет табуляцию в строках
	frame=single,              % рисовать рамку вокруг кода
	tabsize=2,                 % размер табуляции по умолчанию равен 2 пробелам
	captionpos=t,              % позиция заголовка вверху [t] или внизу [b] 
	breaklines=true,           % автоматически переносить строки (да\нет)
	breakatwhitespace=false, % переносить строки только если есть пробел
	escapeinside={\#*}{*)}   % если нужно добавить комментарии в коде
}


\usepackage[left=2cm,right=2cm, top=2cm,bottom=2cm,bindingoffset=0cm]{geometry}
% Для измененных титулов глав:
\usepackage{titlesec, blindtext, color} % подключаем нужные пакеты
\definecolor{gray75}{gray}{0.75} % определяем цвет
\newcommand{\hsp}{\hspace{20pt}} % длина линии в 20pt
% titleformat определяет стиль
\titleformat{\chapter}[hang]{\Huge\bfseries}{\thechapter\hsp\textcolor{gray75}{|}\hsp}{0pt}{\Huge\bfseries}


% plot
\usepackage{pgfplots}
\usepackage{filecontents}
\usetikzlibrary{datavisualization}
\usetikzlibrary{datavisualization.formats.functions}

\begin{document}
	%\def\chaptername{} % убирает "Глава"
	\thispagestyle{empty}
	\begin{titlepage}
		\noindent \begin{minipage}{0.15\textwidth}
			\includegraphics[width=\linewidth]{img/b_logo}
		\end{minipage}
		\noindent\begin{minipage}{0.9\textwidth}\centering
			\textbf{Министерство науки и высшего образования Российской Федерации}\\
			\textbf{Федеральное государственное бюджетное образовательное учреждение высшего образования}\\
			\textbf{~~~«Московский государственный технический университет имени Н.Э.~Баумана}\\
			\textbf{(национальный исследовательский университет)»}\\
			\textbf{(МГТУ им. Н.Э.~Баумана)}
		\end{minipage}
		
		\noindent\rule{18cm}{3pt}
		\newline\newline
		\noindent ФАКУЛЬТЕТ $\underline{\text{«Информатика и системы управления»}}$ \newline\newline
		\noindent КАФЕДРА $\underline{\text{«Программное обеспечение ЭВМ и информационные технологии»}}$\newline\newline\newline\newline\newline
		
		\begin{center}
			\noindent\begin{minipage}{1.1\textwidth}\centering
				\Large\textbf{  Отчет по лабораторной работе №4}\newline
				\textbf{по дисциплине <<Моделирование>>}\newline\newline
			\end{minipage}
		\end{center}
		
		\noindent\textbf{Тема} $\underline{\text{Модели на основе ДУ в частных производных с краевыми условиями 2 и 3 рода}}$\newline\newline
		\noindent\textbf{Студент} $\underline{\text{Романов А.В.~~~~~~~~~~~~~~~~~~~~~~~~~~~~~~~~~~~~~~~~~~~~~~~~~~~~~~~~~~~~~~~~~~~~~~~~~~~~~~~~~~~~~~~~}}$\newline\newline
		\noindent\textbf{Группа} $\underline{\text{ИУ7-63Б~~~~~~~~~~~~~~~~~~~~~~~~~~~~~~~~~~~~~~~~~~~~~~~~~~~~~~~~~~~~~~~~~~~~~~~~~~~~~~~~~~~~~~~~~~~~~~~~}}$\newline\newline
		\noindent\textbf{Оценка (баллы)} $\underline{\text{~~~~~~~~~~~~~~~~~~~~~~~~~~~~~~~~~~~~~~~~~~~~~~~~~~~~~~~~~~~~~~~~~~~~~~~~~~~~~~~~~~~~~~~~~~~~~~~}}$\newline\newline
		\noindent\textbf{Преподаватель} $\underline{\text{Градов В. М.~~~~~~~~~~~~~~~~~~~~~~~~~~~~~~~~~~~~~~~~~~~~~~~~~~~~~~~~~~~~~~~~~~~~~~~~~~~~~~}}$\newline\newline\newline
		
	\begin{center}
		\vfill
		Москва~---~\the\year
		~г.
	\end{center}
\end{titlepage}
	

\section*{Тема работы}
Программно-алгоритмическая реализация моделей на основе дифференциальных уравнений в частных производных с краевыми условиями IIи III рода.\par

\section*{Цель работы}
Получение навыков разработки алгоритмов решения смешанной краевой задачи при реализации моделей, построенных на квазилинейном уравнении параболического типа.\par

\section*{Теоретические сведения}

Задана математическая модель:

\begin{equation}
	c(T) \frac{\partial T}{\partial t} = \frac{\partial}{\partial x} (k(T) \frac{\partial T}{\partial x}) - \frac{2}{R}\alpha(x)T + \frac{2T_{0}}{R}\alpha(x)
\end{equation}\\

Краевые условия:

\begin{equation}
	\begin{cases} t = 0, T(x, 0) = T_{0}
	\\ x = 0, - k(T(0)) \frac{\partial T}{\partial x} = F_{0}
	\\ x = l, - k(T(l)) \frac{\partial T}{\partial x} = \alpha_{N} (T(l) - T_{0})
	\end{cases}
\end{equation}\\

В обозначениях уравнения лекции:

\begin{equation}
	p(x) = \frac{2}{R} \alpha(x)
\end{equation}

\begin{equation}
	f(u) = f(x) = \frac{2T_{0}}{R}\alpha(x)
\end{equation}

Разностная схема с разностным краевым условие при $x = 0$:

\begin{equation*}
	(\frac{h}{8} \buildrel\,\,\frown\over{c_{\frac{1}{2}}} + \frac{h}{4} \buildrel\,\,\frown\over{c_{0}} + \buildrel\,\,\frown\over{X_{\frac{1}{2}}} \frac{\tau}{h} + \frac{\tau h}{8} p_{\frac{1}{2}} + \frac{\tau h}{4}p_{0}) \buildrel\,\,\frown\over{y_{0}} + (\frac{h}{8} \buildrel\,\,\frown\over{c_{\frac{1}{2}}} - \buildrel\,\,\frown\over{X_{\frac{1}{2}}} \frac{\tau}{h} + \frac{\tau h}{8} p_\frac{1}{2}) \buildrel\,\,\frown\over{y_{1}} = 
\end{equation*}

\begin{equation}
	= \frac{h}{8} \buildrel\,\,\frown\over{c_{\frac{1}{2}}} (y_{0} + y_{1}) + \frac{h}{4} \buildrel\,\,\frown\over{c_{0}} y_{0} + \buildrel\,\,\frown\over{F}\tau + \frac{\tau h}{4} (\buildrel\,\,\frown\over{f_{\frac{1}{2}}} + \buildrel\,\,\frown\over{c_{0}})
\end{equation}\\

При получении разностного аналога краевого условия при $x = l$ учесть, что поток:

\begin{equation}
	F_{N} = \alpha{N}(y_{N} - T_{0}), F_{N - \frac{1}{2}} = X_{N - \frac{1}{2}} \frac{y_{N - 1} - y_{N}}{h}
\end{equation}

Заданы начальные параметры:\\

\begin{itemize}
	\item $k(T) = a_{1}(b_{1} + c_{1}T^{m_{1}})$, Вт/см K
	\item $c(T) = a_{2} + b_{2} T^{m_{2}} - \frac{c_{2}}{T^2}$, Дж/c$m^3$ K
	\item $a_{1} = 0.0134, b_{1} = 1, c_{1} = 4.35 \cdot 10^{-4}, m_{1} = 1$
	\item $a_{2} = 2.049, b_{2} = 0.563 \cdot 10^{-3}, c_{2} = 0.528 \cdot 10^{5}, m_{2} = 1$
	\item $\alpha(x) = \frac{c}{x - d}, \alpha_{0} = 0.05$ Вт/с$m^2$ K, $\alpha_{N} = 0.01$ Вт/c$m^2$ K
	\item $l = 10$ см
	\item $T_{0} = 300K$
	\item $R$ = 0.5 см
	\item $F(t) = 50$ Вт/с$m^2$
\end{itemize}

\section*{Исходный код алгоритма}

\begin{lstlisting}[label=picard,caption=, language=python]
from collections import namedtuple
from math import pow, fabs

Params = namedtuple('Params', 'a1 b1 c1 m1 a2 b2 c2 m2 alpha0 alphaN l T0 R F0 h t eps')

params = Params(
	0.0134, 1, 4.35e-4, 1, 2.049, 0.563e-3, 0.528e5, 1, 0.05, 0.01, 10, 300, 0.5, 50, 1e-3, 1, 1e-2
)


def approximation_plus(func, n, step):
	return (func(n) + func(n + step)) / 2


def approximation_minus(func, n, step):
	return (func(n) + func(n - step)) / 2


def k(T):
	return params.a1 * (params.b1 + params.c1 * pow(T, params.m1))


def c(T):
	return params.a2 + params.b2 * pow(T, params.m2) - (params.c2 / pow(T, 2))


def alpha(x):
	d = (params.alphaN * params.l) / (params.alphaN - params.alpha0)
	c = -params.alpha0 * d
	return c / (x - d)


def p(x) :
	return alpha(x) * 2 / params.R


def f(x):
	return alpha(x) * 2 * params.T0 / params.R


def A(T):
	return params.t / params.h * approximation_minus(k, T, params.t)


def D(T):
	return params.t / params.h * approximation_plus(k, T, params.t)


def B(x, T):
	return A(T) + D(T) + params.h * c(T) + params.h * params.t * p(x)


def F(x, T):
	return params.h * params.t * f(x) + T * params.h * c(T)


def get_left_conditions(T):
	c_plus = approximation_plus(c, T[0], params.t)
	k_plus = approximation_plus(k, T[0], params.t)

	K0 = params.h / 8 * c_plus + params.h / 4 * c(T[0]) + params.t / params.h * k_plus + \
		params.t * params.h / 8 * p(params.h / 2) + params.t * params.h / 4 * p(0)

	M0 = params.h / 8 * c_plus - params.t / params.h * k_plus + params.t * params.h / 8 * p(params.h / 2)

	P0 = params.h / 8 * c_plus * (T[0] + T[1]) + params.h / 4 * c(T[0]) * T[0] + \
		params.F0 * params.t + params.t * params.h / 8 * (3 * f(0) + f(params.h))

	return K0, M0, P0


def get_right_conditions(T):
	c_minus = approximation_minus(c, T[-1], params.t)
	k_minus = approximation_minus(k, T[-1], params.t)

	KN = params.h / 8 * c_minus + params.h / 4 * c(T[-1]) + params.t / params.h * k_minus + \
		params.t * params.alphaN + params.t * params.h/ 8 * p(params.l - params.h / 2) + \
		params.t * params.h / 4 * p(params.l)

	MN = params.h / 8 * c_minus - params.t / params.h * k_minus + \
		params.t * params.h / 8 * p(params.l - params.h / 2)

	PN = params.h / 8 * c_minus * (T[-1] + T[-2]) + params.h / 4 * c(T[-1]) * T[-1] + \
		params.t * params.alphaN * params.T0 + params.t * params.h / 4 * (f(params.l) + f(params.l - params.h / 2))

	return KN, MN, PN


def get_new_T(T):
	K0, M0, P0 = get_left_conditions(T)
	KN, MN, PN = get_right_conditions(T)

	xi = [0, -M0 / K0]
	eta = [0, P0 / K0]

	x = params.h
	n = 1

	while x + params.h < params.l:
		Tn = T[n]
		denominator = (B(x, Tn) - A(Tn) * xi[n])

		next_xi = D(Tn) / denominator
		next_eta = (F(x, Tn) + A(Tn) * eta[n]) / denominator

		xi.append(next_xi)
		eta.append(next_eta)

		n += 1
		x += params.h

	T_new = [0 for _ in range(n + 1)]
	T_new[n] = (PN - MN * eta[n]) / (KN + MN * xi[n])

	for i in range(n - 1, -1, -1):
		T_new[i] = xi[i + 1] * T_new[i + 1] + eta[i + 1]

	return T_new


def simple_iteration_method():
	T = [params.T0 for _ in range(int(params.l / params.h) + 1)]
	T_new = [0 for _ in range(int(params.l / params.h) + 1)]

	result = [T]
	ti = 0

	epsilon_condition = True
	while epsilon_condition:
		T_prev = T
		current_max = 1
	
		while current_max >= 1:
			T_new = get_new_T(T_prev)
			current_max = fabs((T[0] - T_new[0]) / T_new[0])
		
			for T_i, Tnew_i in zip(T, T_new):
				d = fabs(T_i - Tnew_i) / Tnew_i
				if d > current_max:
					current_max = d
	
			T_prev = T_new 
	
		result.append(T_new)
		ti += params.t

		epsilon_condition = False
		for T_i, Tnew_i in zip(T, T_new):
			if fabs((T_i - Tnew_i) / Tnew_i) > params.eps:
				epsilon_condition = True

		T = T_new 

	return result, ti
\end{lstlisting}

\section*{Результаты работы программы}

\textbf{1. Представить разностный аналог краевого условия при $x = l$ и его краткий вывод интегро-интерполяционным методом.}\\

Проинтегрируем уравнение на отрезке [$X_{n - \frac{1}{2}}; x_{n}$] (с учётом \textbf{(6)}). Примем $F = -k(u) \frac{\partial T}{\partial x}$.

\begin{equation}
	\int^{x_{N}}_{x_{N-\frac{1}{2}}} int^{t_{m + 1}}_{t_{m}} c(t) \frac{\partial T}{\partial t}dt = - \int^{t_{m + 1}}_{t_{m}}dt \int^{x_{N}}_{x_{N-\frac{1}{2}}} \frac{\partial F}{\partial x} dx - \int^{x_{N}}_{x_{N-\frac{1}{2}}} dx \int^{t_{m + 1}}{t_{m}}p(x)Tdt + \int^{x_{N}}_{x_{N-\frac{1}{2}}} dx \int^{t_{m + 1}}{t_{m}} f(x)dt
\end{equation}\\

Интегрируя аналогично разностному аналогу краевого условия при $x = 0$ (из лекции) получим, учтя \textbf{(6)}:

\begin{equation*}
	\frac{h}{4} (\buildrel\,\,\frown\over{c_{N}}(\buildrel\,\,\frown\over{y_{N}} - y_{N}) - \buildrel\,\,\frown\over{c_{N - \frac{1}{2}}} (\frac{\buildrel\,\,\frown\over{y_{N}} + \buildrel\,\,\frown\over{y_{N - 1}}}{2} - \frac{y_{N} + y_{N + 1}}{2})) =
\end{equation*}

\begin{equation}
	= \tau (\alpha_{N} (\buildrel\,\,\frown\over{y_{N}} - T_{0}) - \buildrel\,\,\frown\over{X_{N}} \frac{\buildrel\,\,\frown\over{y_{N}} + \buildrel\,\,\frown\over{y_{N - 1}}}{h}) - \tau \frac{h}{4} (p_{N}\buildrel\,\,\frown\over{y_{N}} - p_{n - \frac{1}{2} } - \frac{\buildrel\,\,\frown\over{y_{N}} + \buildrel\,\,\frown\over{y_{N - 1}}}{2} + (\buildrel\,\,\frown\over{f_{N}} - \buildrel\,\,\frown\over{f_{N - \frac{1}{2}}}))
\end{equation}\\

Приведем уравнение к виду $K_{N} \buildrel\,\,\frown\over{y_{N}} + M_{N} \buildrel\,\,\frown\over {y_{N - 1}} = P_{N}$:

\begin{equation*}
	(\frac{h}{4} \buildrel\,\,\frown\over{c_{N}} + \frac{h}{8} \buildrel\,\,\frown\over{c_{N - \frac{1}{2}}} + \tau \alpha_{N} + \frac{\tau}{h} \buildrel\,\,\frown\over{X_{N - \frac{1}{2}}} + \frac{h}{4} \tau p_{N} + \frac{h}{8} \tau p_{N - \frac{1}{2}}) \buildrel\,\,\frown\over{y_{n}} + (\frac{h}{8} \buildrel\,\,\frown\over{c_{N - \frac{1}{2}}} - \frac{\tau}{h} \buildrel\,\,\frown\over{X_{N - \frac{1}{2}}} + \frac{h}{8} \tau p_{N - \frac{1}{2}}) \buildrel\,\,\frown\over{y_{N - 1}} =
\end{equation*}

\begin{equation}
	= \alpha_{N} \tau T_{0} + \frac{h}{4} \buildrel\,\,\frown\over{C_{N}}y_{N} + \frac{h}{8} \buildrel\,\,\frown\over{c_{N - \frac{1}{2}}} (y_{N} + y_{N - 1}) + \frac{h}{4} \tau (\buildrel\,\,\frown\over{f_{N}} + \buildrel\,\,\frown\over{f_{N - \frac{1}{2}}})
\end{equation}\\

Будеми использовать простую аппроксимацию:

\begin{equation}
	p_{N - \frac{1}{2}} = \frac{p_{N - 1} + p_{N}}{2}
\end{equation}\\

Получим $K_{0}, M_{0}, P_{0}, K_{N}, M_{N}, P_{N}$:

\begin{equation}
	\begin{cases}\buildrel\,\,\frown\over{K_{0}} \buildrel\,\,\frown\over{y_{0}} + \buildrel\,\,\frown\over{M_{0}} \buildrel\,\,\frown\over{y_{1}} = \buildrel\,\,\frown\over{p_{0}}
		\\ \buildrel\,\,\frown\over{A_{n}}\buildrel\,\,\frown\over{y_{n - 1}} - \buildrel\,\,\frown\over{B_{n}} \buildrel\,\,\frown\over{y_{n}} + \buildrel\,\,\frown\over{D_{n}} \buildrel\,\,\frown\over{y_{n + 1}} = - \buildrel\,\,\frown\over{F_{n}}
		\\ \buildrel\,\,\frown\over{K_{n}} \buildrel\,\,\frown\over{y_{N}} + \buildrel\,\,\frown\over{M_{N - 1}} \buildrel\,\,\frown\over{y_{N - 1}} = \buildrel\,\,\frown\over{P_{N}}
	\end{cases}
\end{equation}\\

Систему \textbf{(11)} решим методом итераций ($s$ - номер итерации): 

\begin{equation}
	A^{s - 1}_{n} y^s_{n + 1} - B^{s - 1}_{n} y^{s}_{n} + D^{s - 1}_{n} y^{s}_{n - 1} = -F^{s - 1}_{n}
\end{equation}\\

\clearpage
\textbf{2. График зависимости температуры $T(x, t_{m})$ от координаты $x$ при нескольких фкисированных значених времени $t_{m}$ при заданных выше параметрах. }\\

\begin{figure}[H]
	\begin{center}
		\includegraphics[scale=0.3]{img/1.png}
	\end{center}
	\caption{График зависимости $T(x_{n}, t)$ от координаты $x$}
	\label{img:1}
\end{figure}

\textbf{3. График зависимости $T(x_{n}, t)$ при нескольких фиксированных значения координаты $x_{n}$}\\

На рисунке \ref{img:2} верхний график соответствует случаю $x = 0$, нижний случаю $x = l$.

\begin{figure}[H]
	\begin{center}
		\includegraphics[scale=0.3]{img/2.png}
	\end{center}
	\caption{График зависимости $T(x_{n}, t)$ при нескольких фиксированных значениях координаты $x_{n}$}
	\label{img:2}
\end{figure}

\clearpage
\section*{Ответы на вопросы}

\textbf{1. Приведите  результаты тестирования  программы (графики, общие соображения, качественный анализ)}\\

При отрицательном тепловом потоке слева идет съем тепла (рисунки 3 - 4).

\begin{figure}[H]
	\begin{center}
		\includegraphics[scale=0.3]{img/3.png}
	\end{center}
	\caption{График зависимости $T(x_{n}, t)$ от координаты $x$ при $F_{0} = -10$}
	\label{img:3}
\end{figure}

\begin{figure}[H]
	\begin{center}
		\includegraphics[scale=0.3]{img/4.png}
	\end{center}
	\caption{График зависимости $T(x_{n}, t)$ при нескольких фиксированных значения координаты $x_{n}$ и $F_{0} = -10$}
	\label{img:4}
\end{figure}

Если обнулить поток $F_0 (T)$, то на выходе должны получить график температуры, установившейся в соответствии с температурой окружающей среды (рисунок 5).


\begin{figure}[H]
	\begin{center}
		\includegraphics[scale=0.4]{img/5.png}
	\end{center}
	\caption{График зависимости $T(x_{n}, t)$ при нулевом потоке.}
	\label{img:5}
\end{figure}


\bibliographystyle{utf8gost705u}  % стилевой файл для оформления по ГОСТу
	
\bibliography{51-biblio}          % имя библиографической базы (bib-файла)
	
	
\end{document}
